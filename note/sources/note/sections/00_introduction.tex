\sectioncentered*{Введение}
\addcontentsline{toc}{section}{Введение}

С течением времени все больше систем начали становиться более комплексными и сложными в реализации. Изначально процесс создания любой информационной системы предполагал собой последовательную поэтапную разработку, которая имеет однонаправленный характер. Из-за этого возникла необходимость в рассмотрении новых подходов для более эффективного проектирования и разработки системы. В настоящее время наиболее перспективным можно выделить объектно-ориентированный подход проектирования. 

Отличительной чертой модели объектно-ориентированного проектирования является отсутствие строгой последовательности в выполнении стадий как в прямом, так и в обратном направлениях процесса проектирования по отдельным компонентам.

Основное преимущество объектно-ориентированного подхода состоит в упрощении проектирования информационной системы при наличии типовых проектных решений по отдельным компонентам. Также наличие инкапсуляции, наследования и полиморфизма у объекта позволяет его легко модифицировать, поскольку модификация касается лишь отдельных компонент из-за их параллельности и автономности. 

Объектно-ориентированные системы можно рассматривать как совокупность независимых объектов. При изменении реализации какого-нибудь объекта или же при добавлении этому объекту неких иных функций другие объекты системы не будут подвержены случайным изменениям. Темой курсового проекта является информационная система вокзала, которая позволит облегчить получение информации о поездах и поездках людей.

Целью курсового проекта является разработка иерархии классов с включенными в них полями и методами для обработки атрибутов, которая будет вспомогательным элементов для взаимодействия пользователя с данными вокзала. Также целью проекта является развитие навыков самостоятельной и творческой работы и закрепление навыков работы на языке \textit{С++}.

Основными задачами данной работы являются – разработка структуры иерархии классов по заданной тематике, разработка схемы алгоритмов реализации отдельных компонентов, проведение тестирования приложении.