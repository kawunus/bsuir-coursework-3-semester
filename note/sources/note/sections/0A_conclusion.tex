\sectioncentered*{Заключение}
\addcontentsline{toc}{section}{Заключение}

В ходе выполнения курсовой работы был разработан проект реляционной базы данных для информационной системы вокзала, направленной на автоматизацию ключевых процессов.

В результате анализа предметной области были определены основные сущности и взаимосвязи между ними, что позволило создать структуру базы данных, включающую таблицы для управления информацией о станциях, маршрутах, поездах, вагонах, местах, билетах и пользователях. Каждая таблица реализует свои уникальные функции: хранение данных о рейсах, статусах мест, расписаниях, продаже билетов и данных пользователей.

Для взаимодействия с каждой таблицей разработаны функции создания, удаления и отображения данных. Это обеспечивает удобное управление базой данных через консольное приложение. В рамках системы реализовано разграничение прав доступа, что позволяет разделять пользователей на обычных и администраторов. Администраторы имеют расширенные возможности, такие как управление поездами, вагонами, расписанием и пользователями, тогда как обычные пользователи могут, например, просматривать расписание и приобретать билеты.

В ходе реализации программы использовались подходы объектно-
ориентированного программирования, включая иерархию классов, детально описанную с определением их полей и методов.